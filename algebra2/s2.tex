\documentclass[10pt,letterpaper]{article}

\usepackage[usenames]{xcolor}
\usepackage[margin=.5in]{geometry}
%\usepackage{xfrac}
\usepackage{polynom}
\usepackage{tikz}
\usepackage{amsmath}
\usepackage{fancyhdr}
\usepackage{datetime}

\pagestyle{fancy}
\renewcommand{\headrulewidth}{0pt}
\setlength{\headheight}{15pt}
\setlength{\headsep}{5pt}
\setlength{\voffset}{-.6in}
\setlength{\topmargin}{0pt}
\renewcommand{\timeseparator}{}
\newdateformat{timestamp}{\THEYEAR\twodigit{\THEMONTH}\twodigit{\THEDAY}}
\fancyhead[C]{\texttt{github.com/Sammidysam/formulaPage} rev {\timestamp\today}\currenttime}

\usetikzlibrary{calc}
\newcommand{\tikzmark}[1]{\tikz[overlay,remember picture] \node (#1) {};}
\DeclareMathOperator{\cis}{cis}
\newcommand{\jhat}{\hat{\jmath}}
\newcommand{\ihat}{\hat{\imath}}
\newcommand{\proj}[2]{\textrm{proj}_{#1}{#2}}

\begin{document}
%\everymath{\scriptscriptstyle}
\parindent=0pt
\abovedisplayshortskip=0pt
\belowdisplayshortskip=0pt
\abovedisplayskip=0pt
\belowdisplayskip=0pt
\fbox{ \begin{minipage}{140pt}
    \centering Positive Trigonometric Functions

    I--All pos. II--sin III--tan IV--cos
\end{minipage}}
\fbox{ \begin{minipage}{96pt}
    \centering Law of Cosines
    \[
    a^2=b^2+c^2-2bc\cdot\cos A
    \]
\end{minipage}}
\fbox{\begin{minipage}{117pt}
    \centering Difference of Cubes
    \[
    a^3\pm b^3=(a\pm b)(a^2\mp ab+b^2)
    \]
\end{minipage}}
\fbox{ \begin{minipage}{31pt}
    \centering Arc Length
    \[
    s=r\theta
    \]
\end{minipage}}

\fbox{ \begin{minipage}{124pt}
    \centering Heron's Formula
    \[
    A=\sqrt{s(s-a)(s-b)(s-c)}
    \]
\end{minipage}}
\fbox{ \begin{minipage}{70pt}
    \centering Change of Base
    \[
    \log_bm=\frac{\log m}{\log b}
    \]
\end{minipage}}
\fbox{ \begin{minipage}{123pt}
    \centering Choose Formula
    \[
    C(x,y)= {\binom{x}{y}}=\frac{x!}{y!(x-y)!}
    \]
\end{minipage}}
\fbox{ \begin{minipage}{100pt}
    \centering Law of Sines
    \[
    \frac{\sin A}{a} = \frac{\sin B}{b} = \frac{\sin C}{c}
    \]
\end{minipage}}

\fbox{ \begin{minipage}{84pt}
    \centering Degrees to Radians
    \[
    \frac{A\cdot\pi}{180}=\theta
    \]
\end{minipage}}
\fbox{ \begin{minipage}{52pt}
    \centering Sector Area
    \[
    A=\frac12r^2\theta
    \]
\end{minipage}}
\fbox{\begin{minipage}{79pt}
    \centering Area of $\Delta$
    \[
    Area = ab\cdot \frac12 \sin C
    \]
\end{minipage}}
\fbox{\begin{minipage}{63pt}
    \centering Polar to $(x,y)$
    \[
    r^2=x^2+y^2
    \] \[
    \tan\theta=\frac yx
    \]
\end{minipage}}
\fbox{ \begin{minipage}{140pt}
    \[
    (\log_ab)(\log_cd)=(\log_ad)(\log_cb)
    \]
\end{minipage}}

\fbox{ \begin{minipage}{65pt}
    \[
    z=a+bi
    \] \[
    |z|=\sqrt{a^2+b^2}
    \]
\end{minipage}}
\fbox{ \begin{minipage}{66pt}
    \[
    z=r\cis\theta
    \] \[
    z^n=r^n\cis(n\theta)
    \]
\end{minipage}}
\fbox{ \begin{minipage}{110pt}
    \centering $n$th roots of $z=r\cis\theta$
    \[
    w_k=r^{1/n}\cis\left(\frac{\theta+2k\pi}{n}\right)
    \]
\end{minipage}}
\fbox{ \begin{minipage}{97pt}
    \[
    z_1z_2=r_1r_2\cis(\theta_1+\theta_2)
    \] \[
    \frac{z_1}{z_2}=\frac{r_1}{r_1}\cis(\theta_1-\theta_2)
    \]
\end{minipage}}
\fbox{ \begin{minipage}{79pt}
    \[
    \vec{v}=\langle a,b\rangle=a\ihat+b\jhat
    \] \[
    |\vec v| = \sqrt{a^2 + b^2}
    \]
\end{minipage}}
\fbox{ \begin{minipage}{51pt}
    \[
    |c\vec u| = |c| |\vec u|
    \]
\end{minipage}}

\fbox{ \begin{minipage}{72pt}
    \centering Dot Product
    \[
    \vec u\cdot\vec v=a_1a_2+b_1b_2
    \]
\end{minipage}}
\fbox{ \begin{minipage}{71pt}
    \centering Dot Product Theorem
    \[
    \vec u\cdot\vec v=|\vec u| |\vec v| \cos\theta
    \]
\end{minipage}}
\fbox{ \begin{minipage}{72pt}
    \centering $\theta$ between $\vec u\ \&\ \vec v$
    \[
    \cos\theta=\frac{\vec u\cdot\vec v}{|\vec u| |\vec v|}
    \]
\end{minipage}}
\fbox{ \begin{minipage}{60pt}
    \centering $\vec u$ and $\vec v$ are prependicular
    \[
    \vec u\cdot\vec v=0
    \]
\end{minipage}}
\fbox{ \begin{minipage}{53pt}
    \centering Component of $\vec u$ along $\vec v$
    \[
    (\vec u\cdot\vec v)/|\vec v|
    \]
\end{minipage}}
\fbox{ \begin{minipage}{85pt}
    \[
    \proj{\vec v}{\vec u}=\left(\frac{\vec u\cdot \vec v}{|\vec v|^2}\right)\vec v
    \]
\end{minipage}}
\fbox{ \begin{minipage}{44pt}
    \centering Work
    \[
    W = \vec F\cdot \vec D
    \]
\end{minipage}}

\fbox{ \begin{minipage}{7.36in}
    \centering Trig Identities

    \fbox{\( \sin^2+\cos^2=1 \)}
    \fbox{\( \tan^2 + 1 = \sec^2 \)}
    \fbox{\( 1+\cot^2=\csc^2 \)}
    \fbox{\( 2\sin u \cos u=\sin(2u) \)}
    \fbox{\( \cos^2u-\sin^2u=\cos(2u) \)}
    \fbox{\( \frac{2\tan u}{1-\tan^2u}=\tan(2u) \)}
    \fbox{\( \sin u\cos v\pm\cos u\sin v=\sin(u\pm v) \)}
    \fbox{\( \cos u\cos v\mp\sin u\sin v=\cos(u\pm v) \)}
    \fbox{\( \frac{\tan u\pm\tan v}{1\mp\tan u\tan v}=\tan(u\pm v) \)}
    \fbox{\( \cot = \frac1{\tan} \)}
    \fbox{\( \csc = \frac1{\sin} \)}
    \fbox{\( \sec = \frac1{\cos} \)}
    \fbox{\( \sin \left(\frac\pi 2-u\right)=\cos u \)}
    \fbox{\( \tan \left(\frac\pi 2-u\right)=\cot u \)}
    \fbox{\( \sec \left(\frac\pi 2-u\right)=\csc u \)}
    \fbox{\( \cos \left(\frac\pi 2-u\right)=\sin u \)}
    \fbox{\( \cot \left(\frac\pi 2-u\right)=\tan u \)}
    \fbox{\( \csc \left(\frac\pi 2-u\right)=\sec u \)}
    \fbox{\( \frac{1-\cos 2x}{2}=\sin^2x \)}
    \fbox{\( \frac{1+\cos 2x}{2}=\cos^2x \)}
    \fbox{\( \frac{1-\cos 2x}{1+\cos 2x}=\tan^2x \)}
    \fbox{\( \pm\sqrt{\frac{1-\cos u}{2}}=\sin \frac u2 \)}
    \fbox{\( \pm\sqrt{\frac{1+\cos u}{2}}=\cos \frac u2 \)}
    \fbox{\( \frac{1-\cos u}{\sin u}=\frac{\sin u}{1+\cos u}=\tan\frac u2 \)}
    \fbox{\( 2\sin\frac{x\pm y}{2}\cos\frac{x\mp y}{2}=\sin x\pm\sin y \)}
    \fbox{\( 2\cos\frac{x+y}{2}\cos\frac{x-y}{2}=\cos x+\cos y \)}
    \fbox{\( -2\sin\frac{x+y}{2}\sin\frac{x-y}{2}=\cos x-\cos y \)}
    \fbox{\( \sin u\cos v=\frac12[\sin(u+v)+\sin(u-v)] \)}
    \fbox{\( \cos u\sin v=\frac12[\sin(u+v)-\sin(u-v)] \)}
    \fbox{\( \cos u\cos v=\frac12[\cos(u+v)+\cos(u-v)] \)}
    \fbox{\( \sin u\sin v=\frac12[\cos(u+v)-\cos(u-v)] \)}
\end{minipage}}

\fbox{\begin{minipage}{89pt}
    \centering Row-Echelon Form
    \[\left[
      \begin{array}{rrrr}
        1 & 2 &-1 & 1 \\
        0 & 1 & 4 &-7 \\
        0 & 0 & 1 &-2 \\
      \end{array}
      \right]\]
\end{minipage}}
\fbox{\begin{minipage}{89pt}
    \centering Reduced Row-Echelon Form
    \[\left[
      \begin{array}{rrrr}
        1 & 0 & 0 &-3 \\
        0 & 1 & 0 & 1 \\
        0 & 0 & 1 &-2 \\
      \end{array}
      \right]\]
\end{minipage}}
\fbox{\begin{minipage}{308pt}
    \centering Using matrix inverses ($AX = B \Rightarrow X = A^{-1}B$)
    \[
    \left[
      \begin{array}{rr}
        2 & -5 \\
        3 & -6 \\
      \end{array}
      \right]
    \left[
      \begin{array}{c}
        x \\
        y \\
      \end{array}
      \right]
    =
    \left[
      \begin{array}{r}
        15 \\
        36 \\
      \end{array}
      \right]
    \Longrightarrow
    \left[
      \begin{array}{c}
        x \\
        y \\
      \end{array}
      \right]
    =
    \left[
      \begin{array}{rr}
        -2 & \frac53 \\
        -1 & \frac23 \\
      \end{array}
      \right]
    \left[
      \begin{array}{r}
        15 \\
        36 \\
      \end{array}
      \right]
    =
    \left[
      \begin{array}{r}
        30 \\
        9 \\
      \end{array}
      \right]
    \]
\end{minipage}}

\fbox{\begin{minipage}{462pt}
    \centering Matrix Multiplication!
    \[
    \left[\begin{array}{rr}
        1 & 3 \\
        -1 & 0 \\
      \end{array}\right]
    \left[\begin{array}{rrr}
        -1 & 5 & 2 \\
        0 & 4 & 7 \\
      \end{array}\right]
    =
    \left[\begin{array}{rrr}
        1\cdot (-1) + 3\cdot 0 & 1\cdot 5 + 3\cdot 4 & 1 \cdot 2 + 3\cdot 7 \\
        (-1)\cdot(-1) + 0\cdot 0 & (-1)\cdot5+0\cdot4 & (-1)\cdot2+0\cdot7 \\
      \end{array}\right]
    =
    \left[\begin{array}{rrr}
        -1 & 17 & 23 \\
        1 & -5 & -2 \\
      \end{array}\right]
    \]
\end{minipage}}

\fbox{\begin{minipage}{225pt}
    \centering $2\times2$ Matrix Inverse
    \[
    \textrm{If }A=
    \left[\begin{array}{rr}
        a & b \\
        c & d \\
      \end{array}\right]
    \textrm{, then }
    A^{-1}=\frac{1}{ad-bc}
    \left[\begin{array}{rr}
        d & -b \\
        -c & a \\
      \end{array}\right]
    \]
\end{minipage}}
\fbox{\begin{minipage}{276pt}
    \centering $n\times n$ Matrix Inverse
    \[
    \left[\begin{array}{rrr}
        1 & -2 & -4 \\
        2 & -3 & -6 \\
        -3 & 6 & 15 \\
      \end{array}
      \middle|
      \begin{array}{rrr}
        1 & 0 & 0 \\
        0 & 1 & 0 \\
        0 & 0 & 1 \\
      \end{array}
      \right]
    \longrightarrow
    \left[\begin{array}{rrr}
        1 & 0 & 0 \\
        0 & 1 & 0 \\
        0 & 0 & 1 \\
      \end{array}
      \middle|
      \begin{array}{rrr}
        -3 & 2 & 0 \\
        -4 & 1 & -\frac23 \\
        1 & 0 & \frac12
      \end{array}
      \right]
    \]
\end{minipage}}

\fbox{\begin{minipage}{147pt}
    \centering $2\times2$ Matrix Determinant
    \[
    \det(A) = |A| =
    \left|\begin{array}{cc}
    a & b \\
    c & d \\
    \end{array}\right|
    = ad - bc
    \]
\end{minipage}}
\fbox{\begin{minipage}{145pt}
    \centering Minor $M_{ij}$:
    Take the matrix and delete the $i$th row and the $j$th column. Find the determinant
\end{minipage}}
\fbox{\begin{minipage}{58pt}
    \centering Cofactor $A_{ij}$
    \[
    (-1)^{i+j}M_{ij}
    \]
\end{minipage}}

\fbox{\begin{minipage}{320pt}
    \centering $n\times n$ Matrix Determinant (can move along any row/column)
    \[
    \det(A) = |A| =
    \left|\begin{array}{cccc}
    \color{red}a_{11} & \color{red}a_{12} & \color{red}\cdots & \color{red}a_{1n} \\
    a_{21} & a_{22} & \cdots & a_{2n} \\
    \vdots & \vdots & \ddots & \vdots \\
    a_{m1} & a_{m2} & \cdots & a_{mn} \\
    \end{array}\right|
    = {\color{red}a_{11}}A_{11} + {\color{red}a_{12}}A_{12} + \cdots + {\color{red}a_{1n}}A_{1n}
    \]
\end{minipage}}
\fbox{ \begin{minipage}{115pt}
    \centering Common Sums
    \[
    \sum_{k=1}^n c = nc
    \] \[
    \sum_{k=1}^n k = \frac{n(n+1)}{2}
    \] \[
    \sum_{k=1}^n k^2 = \frac{n(n+1)(2n+1)}{6}
    \] \[
    \sum_{k=1}^n k^3 = \frac{n^2(n+1)^2}{4}
    \]
\end{minipage}}
\newpage
\fbox{\begin{minipage}{266pt}
    \centering Algebra of Functions

    \raggedright Let $f$ and $g$ be functions with domains $A$ and $B$.
    \[
    \begin{array}{cl}
      (f+g)(x)=f(x)+g(x) & \textrm{Domain~} A \cap B  \\
      (f-g)(x)=f(x)-g(x) & \textrm{Domain~} A \cap B \\
      (fg)(x)=f(x)g(x) & \textrm{Domain~} A \cap B \\
      \left(\frac{f}{g}\right)(x)=\frac{f(x)}{g(x)} & \textrm{Domain~} \{x \in A \cap B \mid g(x) \neq 0\} \\
      (f \circ g)(x)=f(g(x)) & \textrm{Domain~} \{x \in B \mid g(x) \in A\}
    \end{array}
    \]
\end{minipage}}
\fbox{\begin{minipage}{133pt}
    \centering Polynomial Synthetic Division
    \[
    (x^3+x^2-1) \div (x+2)
    \]
    \polyset{showbase=top}
    \hspace{-7pt}\polyhornerscheme[x=-2,stage=8]{x^3+x^2-1}

    Result is $x^2 - x + 2 - \frac{5}{x + 2}$
\end{minipage}}
\fbox{\begin{minipage}{118pt}
    \centering Polynomial Long Division
    \polylongdiv[stage=8]{x^3+x^2-1}{x+2}
\end{minipage}}

\fbox{\begin{minipage}{140pt}
    \centering Rational Roots Theorem
    \[
    2x^3+2x^2-3x-6
    \]
    \hspace{15pt}$\pm1, \pm2$ \hfill $\pm1, \pm2, \pm3, \pm6$

    Possible rational roots:

    $\pm1, \pm\frac{1}{2}, \pm2, \pm3, \pm\frac{3}{2}, \pm6$
\end{minipage}}
\fbox{\begin{minipage}{140pt}
    \centering Decartes' Rule of Signs

    \raggedright Count num. of sign changes
    \[
    P(x)=3x^6+4x^5+\tikzmark{a}3x^3-\tikzmark{b}x-3
    \begin{tikzpicture}[overlay,remember picture,out=315,in=225,distance=0.4cm]
      \draw[-latex] (a) to (b);
    \end{tikzpicture}
    \]
    1 positive real root
    \[
    P(-x)=\tikzmark{a}3x^6-\tikzmark{b}4x^5-\tikzmark{c}3x^3+\tikzmark{d}x-\tikzmark{e}3
    \begin{tikzpicture}[overlay,remember picture,out=315,in=225,distance=0.4cm]
      \draw[-latex] (a) to (b);
      \draw[-latex] (c) to (d);
      \draw[-latex,out=290,in=250] (d) to (e);
    \end{tikzpicture}
    \vspace*{5pt}
    \]
    1 or 3 negative real roots
\end{minipage}}
\fbox{\begin{minipage}{140pt}
    \centering Logarithm Formulas
    \begin{align*}
      \log(m\cdot n) &= \log m + \log n \\
      \log\left(\frac{m}{n}\right) &= \log m - \log n \\
      \log(m^n) &= n \cdot \log m \\
      \log_bb^x &= x =b^{\log_bx}
    \end{align*}
\end{minipage}}
\fbox{\begin{minipage}{67pt}
    \centering Trigonometric Reciprocals
    \begin{align*}
      \cot &= \frac1{\tan} \\
      \csc &= \frac1{\sin} \\
      \sec &= \frac1{\cos}
    \end{align*}
\end{minipage}}

\fbox{\begin{minipage}{216pt}
    \centering Horizontal Asymptotes
    \begin{align*}
      y&=\frac{2x^2-4x+5}{x^2-2x+1} &&\, \textrm{Original Equation}\\
      &=\frac{2x^2}{x^2} &&\, x \to \infty, \textrm{other terms $\to$ tiny}\\
      &=2 &&\, \textrm{Cancel, horizontal asymptote}
    \end{align*}
\end{minipage}}
\fbox{\begin{minipage}{212pt}
    \centering Slant Asymptotes
    \begin{align*}
      y&=\frac{x^2-4x-5}{x-3} &&\, \textrm{Original Equation}\\
      &=x-1-\frac{8}{x-3} &&\, \textrm{Divide}\\
      &=x-1 &&\, x\to\infty, \textrm{other terms $\to$ tiny}
    \end{align*}
\end{minipage}}

\fbox{\begin{minipage}{181pt}
    \centering Vertical Asymptotes
    \begin{align*}
      y&=\frac{2x^2-4x+5}{x^2-2x+1} &&\, \textrm{Original Equation}\\
      &=\frac{2x^2-4+5}{(2x-1)(x+2)} &&\, \textrm{Factor demoniator}\\
      x&=\frac12 \textrm{~or~} x=-2 &&\, \textrm{Impossible}
    \end{align*}
\end{minipage}}
\fbox{\begin{minipage}{152pt}
    \centering End Behavior

    \begin{minipage}{40pt}
      \[
      y=x^n
      \]
      $n$ is even\vspace{5pt}
      \[
      y=x^n
      \]
      $n$ is odd
    \end{minipage}
    \begin{minipage}{30pt}
      \begin{tikzpicture}
        \draw[<->] (-.5,0) -- (.5,0);
        \draw[<->] (0,-.5) -- (0,.5);
        \draw[in=270,out=0] (.25,.25) to (.5,.5);
        \draw[in=270,out=180] (-.25,.25) to (-.5,.5);
      \end{tikzpicture}
      \begin{tikzpicture}
        \draw[<->] (-.5,0) -- (.5,0);
        \draw[<->] (0,-.5) -- (0,.5);
        \draw[in=270,out=0] (.25,.25) to (.5,.5);
        \draw[in=90,out=180] (-.25,-.25) to (-.5,-.5);
      \end{tikzpicture}
    \end{minipage}~
    \begin{minipage}{40pt}
      \[
      y=-x^n
      \]
      $n$ is even\vspace{5pt}
      \[
      y=-x^n
      \]
      $n$ is odd
    \end{minipage}
    \begin{minipage}{30pt}
      \begin{tikzpicture}
        \draw[<->] (-.5,0) -- (.5,0);
        \draw[<->] (0,-.5) -- (0,.5);
        \draw[in=90,out=0] (.25,-.25) to (.5,-.5);
        \draw[in=90,out=180] (-.25,-.25) to (-.5,-.5);
      \end{tikzpicture}
      \begin{tikzpicture}
        \draw[<->] (-.5,0) -- (.5,0);
        \draw[<->] (0,-.5) -- (0,.5);
        \draw[in=90,out=0] (.25,-.25) to (.5,-.5);
        \draw[in=270,out=180] (-.25,.25) to (-.5,.5);
      \end{tikzpicture}
    \end{minipage}
\end{minipage}}
\fbox{\begin{minipage}{119pt}
    \centering $m\times n$ matrix
    \[\left[
      \begin{array}{ccccc}
        a_{11} & a_{12} & \cdots & a_{1n} \\
        a_{21} & a_{22} & \cdots & a_{2n} \\
        \vdots & \vdots & \ddots & \vdots \\
        a_{m1} & a_{m2} & \cdots & a_{mn} \\ % worked out beautifully, huh? Completely unintentional.
      \end{array}
      \right]\]
\end{minipage}}

\fbox{\begin{minipage}{207pt}
    \centering $y=\sin x$ in red; $y=\cos x$ in blue
    \begin{tikzpicture}[yscale=1.5]
      \draw [help lines, ->] (0,0) -- (6.4,0);
      \node [below] at (6.4,0) {$2\pi$};
      \node [below] at (pi,0) {$\pi$};
      \node [left] at (0,1) {1};
      \node [left] at (0,-1) {-1};
      \draw [help lines, <->] (0,-1.1) -- (0,1.1);
      \draw [red,domain=0:2*pi] plot (\x, {sin(\x r)});
      \draw [blue, domain=0:2*pi] plot (\x, {cos(\x r)});

    \end{tikzpicture}
\end{minipage}}
\fbox{ \begin{minipage}{110pt}
    \centering sin/cos Graph Properties

    \raggedright If in form:
    \[
    y=a\sin k(x-b)
    \]
    amplitude  $|a|$, period  $2\pi/k$, phase shift $b$
\end{minipage}}
\fbox{ \begin{minipage}{193pt}
    \centering Allowed row operations
    \begin{enumerate}
    \item Add a multiple of one row to another
    \item Multiply a row by a nonzero constant
    \item Interchange two rows
    \end{enumerate}
\end{minipage}}
\fbox{ \begin{minipage}{250pt}
    \[
    \textrm{If }
    \left\{\begin{array}{l}
    ax + by = r \\
    cx + dy = s \\
    \end{array}\right.
    \textrm{, then }
    x = \frac{\left|\begin{array}{cc}
      r & b \\
      s & d \\
      \end{array}\right|}
    {\left|\begin{array}{cc}
      a & b \\
      c & d \\
      \end{array}\right|}
    \textrm{ and }
    y = \frac{\left|\begin{array}{cc}
      a & r \\
      c & s \\
      \end{array}\right|}
    {\left|\begin{array}{cc}
      a & b \\
      c & d \\
      \end{array}\right|}
    \]
\end{minipage}}
\fbox{ \begin{minipage}{80pt}
    \centering Vertical Parabola
    \[
    x^2 = 4py
    \]
    \raggedright    $V(0,0)$, $F(0,p)$, directrix $y=-p$
\end{minipage}}
\fbox{ \begin{minipage}{106pt}
    \centering Ellipse
    \[
    \frac{x^2}{(a\textrm{ or }b)^2} + \frac{y^2}{(a\textrm{ or }b)^2} = 1
    \] \[
    c^2 = a^2 - b^2
    \]
\end{minipage}}
\fbox{ \begin{minipage}{53pt}
    \centering Eccentricity
    \[
    e = \frac ca,
    \]
\end{minipage}}

\fbox{ \begin{minipage}{90pt}
    \centering Hyperbola
    \[
    \frac{x\textrm{ or }y^2}{a^2} - \frac{x\textrm{ or }y^2}{b^2} = 1
    \] \[
    c^2 = a^2 + b^2
    \]
\end{minipage}}
\fbox{ \begin{minipage}{53pt}
    \centering Shifted Conic
    $V(h,k)$,
    $x$ to $(x-h)$,
    $y$ to $(y-k)$
\end{minipage}}
\fbox{ \begin{minipage}{100pt}
    \centering Polar Conics
    \[
    r = \frac{ed}{1\pm e(\cos\textrm{ or }\sin)\theta}
    \]
\end{minipage}}
\fbox{ \begin{minipage}{133pt}
    \centering Derivative Formula
    \[
    f^{-1} (a) = \lim_{h \to 0} \frac{ f(a+h)-f(a) }{h}
    \]
\end{minipage}}
\fbox{ \begin{minipage}{99pt}
    \centering Area
    \[
    A = \lim_{n \to \infty} \sum_{k=1}^n f(x_k) \Delta x
    \] \[
    \Delta x = \frac{b - a}{n}
    \] \[
    x_k = a + k\Delta x
    \]
\end{minipage}}

\fbox{ \begin{minipage}{80pt}
    \centering Horizontal Parabola
    \[
    y^2 = 4px
    \]
    \raggedright    $V(0,0)$, $F(p,0)$, directrix $x=-p$
\end{minipage}}
\fbox{ \begin{minipage}{70pt}
    \centering Parabolas
    Latus rectum is $\lvert 4p \rvert$
\end{minipage}}
\fbox{ \begin{minipage}{145pt}
    \centering Ellipses
    \[
    a^2 > b^2
    \]
    $x^2$ first of terms means more horizontal,
    major axis length is $2a$,
    minor axis length is $2b$,
    latus rectum is $\frac{2b^2}{a}$,
    foci on major axis $F(\pm c,0)$ or $F(0,\pm c)$
\end{minipage}}
\fbox{ \begin{minipage}{185pt}
    \centering \textbf{Hyperbolas}
    $a^2$ forms positive term with $x$ or $y$,
    horizontal when $x^2$ is first of terms,
    $V(\pm a,0)$ or $V(0,\pm a)$,
    $B(0,\pm b)$ or $B(\pm b,0)$,
    transverse axis length is $2a$,
    conjugate axis length is $2b$,
    asymtote slopes $\pm \frac{b}{a}$ or $\pm \frac{a}{b}$,
    foci on transverse axis $F(\pm c,0)$ or $F(0,\pm c)$,
    latus rectum is $\frac{2b^2}{a}$
\end{minipage}}
\end{document}
