\documentclass[10pt,letterpaper]{article}

\usepackage[usenames]{xcolor}
\usepackage[margin=.5in]{geometry}
%\usepackage{xfrac}
\usepackage{polynom}
\usepackage{tikz}
\usepackage{amsmath}
\usepackage{fancyhdr}
\usepackage{datetime}
\usepackage{siunitx}

\pagestyle{fancy}
\renewcommand{\headrulewidth}{0pt}
\setlength{\headheight}{15pt}
\setlength{\headsep}{5pt}
\setlength{\voffset}{-.6in}
\setlength{\topmargin}{0pt}
\renewcommand{\timeseparator}{}
\newdateformat{timestamp}{\THEYEAR\twodigit{\THEMONTH}\twodigit{\THEDAY}}
\fancyhead[C]{Physics I H Semester 1 rev {\timestamp\today}\currenttime ~\texttt{github.com/cg505/formulaPage}}

\usetikzlibrary{scopes}

%% \usetikzlibrary{calc}
%% \newcommand{\tikzmark}[1]{\tikz[overlay,remember picture] \node (#1) {};}
%% \DeclareMathOperator{\cis}{cis}
%% \newcommand{\jhat}{\hat{\jmath}}
%% \newcommand{\ihat}{\hat{\imath}}
%% \newcommand{\proj}[2]{\textrm{proj}_{#1}{#2}}

\begin{document}
%\everymath{\scriptscriptstyle}
\parindent=0pt
\abovedisplayshortskip=0pt
\belowdisplayshortskip=0pt
\abovedisplayskip=0pt
\belowdisplayskip=0pt

\fbox{ \begin{minipage}{315pt}
\centering \(\Delta x \to \Delta\theta\) displacement, \(v \to \omega\) vel., \(a = \bar{a} \to \alpha\) constant accel., \(\bar{x}\) avg. \(x\)

\fbox{\( v = v_o + at \)}
\fbox{\( \bar{v} = \frac{v_0 + v}{2} \)}
\fbox{\( \Delta x = \bar{v} t \)}
\fbox{\( \Delta x = v_0 t + \frac12 a t^2 \)}
\fbox{\( v^2 = v_0^2 + 2 a \Delta x \)}
\end{minipage}}
%%
\fbox{ \begin{minipage}{60pt}
\centering Earth gravity
\[
g = \SI{9.80}{m/s^2}
\]
\end{minipage}}
%%
\fbox{ \begin{minipage}{114pt}
\centering Effing horses
\[
\SI{1}{hp} = \SI{550}{ft.lb/s} = \SI{746}{W}
\]
\end{minipage}}
%%

\fbox{ \begin{minipage}{36pt}
\centering Newton
\[
F = m a
\]
\end{minipage}}
%%
\fbox{ \begin{minipage}{35pt}
\centering Friction
\[
f = \mu N
\]
\end{minipage}}
%%
\fbox{ \begin{minipage}{42pt}
\centering Static
\[
f_s \leq \mu_s N
\]
\end{minipage}}
%%
\fbox{ \begin{minipage}{38pt}
\centering Weight
\[
W = m g
\]
\end{minipage}}
%%
\fbox{ \begin{minipage}{35pt}
\centering
\[
a_c = \frac{v^2}{r}
\]
\end{minipage}}
%%
\fbox{ \begin{minipage}{50pt}
\centering Momentum
\[
\vec p = m \vec v
\]
\end{minipage}}
%%
\fbox{ \begin{minipage}{103pt}
\centering Impulse
\[
\vec F \Delta t = \Delta \vec p = m \vec v_f - m \vec v_i
\]
\end{minipage}}
%%

\fbox{ \begin{minipage}{121pt}
\centering Universal gravitational constant
\[
G = \SI{6.67e-11}{N.m^2/k.g^2}
\]
\end{minipage}}
%%
\fbox{ \begin{minipage}{50pt}
\centering
\[
v = \sqrt{\frac{G M}{r}}
\]
\end{minipage}}
%%
\fbox{ \begin{minipage}{39pt}
\centering Period
\[
T = \frac{2 \pi r}{v}
\]
\end{minipage}}
%%
\fbox{ \begin{minipage}{53pt}
\centering
\[
T^2 = \frac{4 \pi^2 r^3}{G M}
\]
\end{minipage}}
%%
\fbox{ \begin{minipage}{34pt}
\centering \(T\) in yr, \(r\) in AU
\[
T^2 = r^3
\]
\end{minipage}}
%%
\fbox{ \begin{minipage}{58pt}
\centering
\[
F = \frac{G m_1 m_2}{r^2}
\]
\end{minipage}}
%%

\fbox{ \begin{minipage}{100pt}
\centering Work
\[
W = \vec F \cdot \vec x = \lvert F \rvert \lvert x \rvert \cos \theta
\]
\end{minipage}}
%%
\fbox{ \begin{minipage}{40pt}
\centering Springs
\[
F = -k x
\]
\end{minipage}}
%%
\fbox{ \begin{minipage}{62pt}
\centering Potential
\[
\textsc{pe}_{grav} = m g y
\]
\end{minipage}}
%%
\fbox{ \begin{minipage}{73pt}
\centering Energy
\[
\textsc{pe}_{elastic} = \tfrac12 k x^2
\]
\end{minipage}}
%%
\fbox{ \begin{minipage}{50pt}
\centering Kinetic E.
\[
\textsc{ke} = \tfrac12 m v^2
\]
\end{minipage}}
%%
\fbox{ \begin{minipage}{98pt}
\centering Work and KE
\[
W = \Delta\textsc{ke} = \textsc{ke}_f - \textsc{ke}_i
\]
\end{minipage}}
%%

\fbox{ \begin{minipage}{92pt}
\centering Conservation of E.
\[
\textsc{ke}_i + \textsc{pe}_i = \textsc{ke}_f + \textsc{pe}_f
\]
\end{minipage}}
%%
\fbox{ \begin{minipage}{125pt}
\centering Perfectly inelastic collision
\[
m_1 \vec v_{1_i} + m_2 \vec v_{2_i} = (m_1 + m_2) \vec v_f
\]
\end{minipage}}
%%
\fbox{ \begin{minipage}{135pt}
\centering Inelastic collision (not perf.)
\[
m_1 \vec v_{1_i} + m_2 \vec v_{2_i} = m_1 \vec v_{1_f} + m_2 \vec v_{2_f}
\]
\end{minipage}}
%%
\fbox{ \begin{minipage}{97pt}
\centering Elastic collision
\[
\vec v_{1_i} - \vec v_{2_i} = -(\vec v_{1_f} - \vec v_{2_f})
\]
\end{minipage}}
%%

\fbox{ \begin{minipage}{28pt}
\centering Arc length
\[
s = r \theta
\]
\end{minipage}}
%%
\fbox{ \begin{minipage}{46pt}
\centering Tangential Velocity
\[
v = r \omega
\]
\end{minipage}}
%%
\fbox{ \begin{minipage}{43pt}
\centering Changes speed
\[
a_{tan} = r \alpha
\]
\end{minipage}}
%%
\fbox{ \begin{minipage}{43pt}
\centering \vspace{-6pt}
\[
a_c = a_r =
\]\[
\frac{v^2}{r}= r \omega^2
\]
\end{minipage}}
%%
\fbox{ \begin{minipage}{43pt}
\centering \vspace{-6pt}
\[
a_c = a_r =
\]\[
\frac{v^2}{r}= r \omega^2
\]
\end{minipage}}
%%
\fbox{ \begin{minipage}{102pt}
\centering Torque (in \si{m.N} not \si{J})
\[
\tau = \lvert \vec r \times \vec F \rvert = \lvert \vec r \rvert \lvert \vec F \rvert \sin \theta
\]
\end{minipage}}
%%
\fbox{ \begin{minipage}{52pt}
\centering Rot. inertia
\[
I = \Sigma m r^2
\]
See back
\end{minipage}}
%%
\fbox{ \begin{minipage}{35pt}
\centering \vspace{-6pt}
\[
F = m a
\]\[
\tau = I \alpha
\]
\end{minipage}}
%%

\fbox{ \begin{minipage}{50pt}
\centering Angular momentum
\[
L = I \omega
\]
\end{minipage}}
%%
\fbox{ \begin{minipage}{98pt}
\centering Angular impulse
\[
\tau \Delta t = \Delta L = I \omega_f - I \omega_i
\]
\end{minipage}}
%%
\fbox{ \begin{minipage}{120pt}
\centering Linear kinetic energy
\[
\textsc{ke}_{linear} = \textsc{ke}_{trans} = \frac12 m v^2
\]
\end{minipage}}
%%
\fbox{ \begin{minipage}{60pt}
\centering Rotational
\[
\textsc{ke}_{rot} = \frac12 I \omega^2
\]
\end{minipage}}
%%
\fbox{ \begin{minipage}{50pt}
\centering Used below
\[
\mu = \frac{m}{L}
\]
\end{minipage}}
%%

\fbox{ \begin{minipage}{41pt}
\centering Springs
\[
\omega = \sqrt{\frac{k}{m}}
\]
\end{minipage}}
%%
\fbox{ \begin{minipage}{60pt}
\centering SHM (smh)
\[
x = A \cos(\omega t)
\]\[
\textrm{or~} A \sin(\omega t)
\]
\end{minipage}}
%%
\fbox{ \begin{minipage}{56pt}
\centering Period/freq
\[
f = \frac{1}{T} = \frac{\omega}{2\pi}
\]
\end{minipage}}
%%
\fbox{ \begin{minipage}{51pt}
\centering Pendulum
\[
T = 2 \pi \sqrt{\frac{L}{g}}
\]
\end{minipage}}
%%
\fbox{ \begin{minipage}{62pt}
\centering Sound in air \\
\scriptsize v in \si{m/s},\\
temp T in \si{\celsius} \\
\normalsize
\(
v = 331 + 0.6 T
\)
\end{minipage}}
%%
\fbox{ \begin{minipage}{51pt}
\centering Wavelength
\[
\lambda = \frac{v}{f}
\]
\end{minipage}}
%%
\fbox{ \begin{minipage}{53pt}
\centering String
\[
v = \sqrt{\frac{F_{tens}}{\mu}}
\]
\end{minipage}}
%%
\fbox{ \begin{minipage}{50pt}
\centering Harmonics \vspace{2pt}
\[
f_n = \frac{n v}{2 L}
\]
\end{minipage}}
%%

\fbox{ \begin{minipage}{47pt}
\centering Open tube
\[
f_n = \frac{n v}{2 L}
\]
\end{minipage}}
%%
\fbox{ \begin{minipage}{52pt}
\centering Closed tube
\[
f_n = \frac{n v}{4 L}
\]
\scriptsize n is odd
\end{minipage}}
%%
\fbox{ \begin{minipage}{41pt}
\centering Intensity
\[
I = \frac{P}{4 \pi r^2}
\]
\end{minipage}}
%%
\fbox{ \begin{minipage}{72pt}
\centering Loudness, dB
\[
\beta = 10 \log \left( \frac{I}{I_0} \right)
\]
\end{minipage}}
%%
\fbox{ \begin{minipage}{77pt}
\centering Standard Intensity
\[
I_0 = \SI{e-12}{W/m^2}
\]
\end{minipage}}
%%
\fbox{ \begin{minipage}{81pt}
\centering Doppler Effect
\[
f' = f \left( \frac{v \pm v_{obs}}{v \mp v_{src}} \right)
\]
\end{minipage}}
%%

\newpage
\fbox{\begin{minipage}{142pt}
\centering Values of Rotational Inertia I
\raggedright
\[
\begin{array}{l l}
\textrm{Thin hoop} & m r^2\\
\textrm{Solid cylinder} & \frac12 m r^2\\
\textrm{Hollow cylinder} & \frac12 m(r_1^2 + r_2^2)\\
\textrm{Sphere} & \frac25 m r^2\\
\end{array}
\]
A sphere is the fastest object
\end{minipage}}
%%
\fbox{\begin{minipage}{90pt}
\centering Equilibrium Probz
\begin{align*}
F_{up} &= F_{down} \\
F_{left} &= F_{right} \\
\tau_{CW} &= \tau_{CCW}
\end{align*}
\raggedright Choose pivot point at unknown force
\end{minipage}}
%%

\fbox{\begin{minipage}{143pt}
\centering Free body diagrams!
%% \begin{table}{pp}

\def\iangle{35} % Angle of the inclined plane

\def\down{-90}
\def\arcr{0.5cm} % Radius of the arc used to indicate angles

\begin{tikzpicture}[
    force/.style={>=latex,draw=blue,fill=blue},
    fakeforce/.style={densely dashed,>=latex,draw=blue,fill=blue},
    axis/.style={densely dashed,gray,font=\small},
    M/.style={rectangle,draw,fill=lightgray,minimum size=0.5cm,thin},
    m/.style={rectangle,draw=black,fill=lightgray,minimum size=0.3cm,thin},
    plane/.style={draw=black,fill=blue!10},
    string/.style={draw=red, thick},
    pulley/.style={thick},
]

\matrix[column sep=1cm] {
    %% Sketch
    \draw[plane] (0,-1) coordinate (base)
                     -- coordinate[pos=0.5] (mid) ++(\iangle:3) coordinate (top)
                     |- (base) -- cycle;
    \path (mid) node[M,rotate=\iangle,yshift=0.25cm] (M) {};
    %% \draw[pulley] (top) -- ++(\iangle:0.25) circle (0.25cm)
    %%                ++ (90-\iangle:0.5) coordinate (pulley);
    %% \draw[string,->] (M.east) -- ++(\iangle:1.5cm) {};

    \draw[->] (base)++(\arcr,0) arc (0:\iangle:\arcr);
    \path (base)++(\iangle*0.5:\arcr+5pt) node {$\theta$};
    %%

%% &
    %% Free body diagram of M
    \begin{scope}[rotate=\iangle, shift=(M)]
        %% \node[M,transform shape] (M) {};
        % Draw axes and help lines

        {[axis,->]
            \draw (0,-1) -- (0,2) node[right] {$+y$};
            \draw (M) -- ++(2,0) node[right] {$+x$};
            % Indicate angle. The code is a bit awkward.

            \draw[solid,shorten >=0.5pt] (\down-\iangle:\arcr)
                arc(\down-\iangle:\down:\arcr);
            \node at (\down-0.5*\iangle:1.3*\arcr) {$\theta$};
        }

        % Forces
        {[force,->]
            % Assuming that Mg = 1. The normal force will therefore be cos(alpha)
            \draw (M.north) -- ++(0,{cos(\iangle)}) node[above right] {$N$};
            \draw (M.west) -- ++(-1,0) node[above left] {$f=\mu N$};
            \draw (M.east) -- ++(1,0) node[above] {$T$};
        }

    \end{scope}
    % Draw gravity force. The code is put outside the rotated
    % scope for simplicity. No need to do any angle calculations.
    \draw[force,->] (M.center) -- ++(0,-1) node[below] {$mg$};
    %% \draw[fakeforce,->] (M.center) -- ++(\iangle-90:1) node[above right] {$Mg$};
    %% \draw[fakeforce,->] (M.center)++(\iangle-90:1) -- ++(0,-1) node[below] {$Mg$};
    %%

\\
};
\end{tikzpicture}
%% &
\[
N = m g \cos \theta
\]\[
T - m g \sin \theta - f = ma
\]\[
T - m g \sin \theta - \mu m g \cos \theta = m a
\]
%% \end{table}
\end{minipage}}
\end{document}
